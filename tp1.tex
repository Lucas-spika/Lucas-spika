%%----tipo de documento--------------

\documentclass{article}

%%--Paquetes usados--------------------
\usepackage[utf8]{inputenc}
\usepackage[spanish]{babel}
\usepackage{blindtext} %permite los espacios muertos entre lineas
\usepackage{fullpage}  %cambia a para utilizar toda la pag
\usepackage{setspace} %permite setear el interlineado
\usepackage[a4paper, total={6in, 8in}] {geometry}
\usepackage{changepage} %contiene la libreria para ajustar
\usepackage{graphicx} %permite insertar imagenes
\usepackage{xcolor} %permite colores
%%-----------------------------------------
%%------variables definidas--------------------
\pagestyle{empty} %saca el número a las paginas
\onehalfspace %lo cambia a 1.5

%%------------------------------------------------

\title{Un monológo de fanzine}
\date{}
\author{Lucas Spika}


%%----------Adentro del documento====------------------------------------------------------

\begin{document}





%%-------------Termino de titulo------------------

Sobre la esquina de Bondpland estaba un paraíso medio bajo con un saltito medio alto se llega, siempre que sea más bien tarde-noche así volví a casa, no la mía sino la de mi novia: lleno de paraíso rosando el piso. Entro (me abren en realidad) y veo una disposición conocida, todos escuchando, solo una hablando (un monológo) Esta vez con un fanzine y tienen que ir porque:\\
A. no saben como empieza, esta todo oscuro, no hay nada. Humo  en los pies y se prende todo. Y hay una bici (dada vuelta), un hombre y una mujer pedaleando.\\
B. él es el director aunque también hace de un personaje en la obra y te explica que cuando iba al teatro su viejo siempre decía esta me gustó, pero no la entendí, esta otra no me gusto y no la entendí y por eso el hace de un personaje haciendo de él mismo, el director, pero como personaje. 
Ahí es donde estamos llegando tarde yo y mi novia salimos por Juramneto hasta el subte y después otro subte hasta Parque Patricios.
\\

Él dice Celso y Celia: él pobre, ella una cheta con casa. Son amigos. Ha y pasa en los 90’s y en un verano. Se quedan solos, a Celso lo contratan de geisha (ahora se llama tokyto) cae rendido a los brazos de un hombre con mirada firme para estar toda la vida, camina pensando en abrazarlo, en que lo lleve a dar una vuelta en su moto, siempre hace cinco cuadras para verlo y él con un timbre rizado, algo áspero le dice: hola pa, que onda querés una pitada. Las cuadras las camina porque él es  kioskero. Y dale viejo elegí o raja de acá que te hago boleta, dale me querés matar del sueño, quiero cerrar gato. No te quiero garchar. 
\\

Y esta Celia esperando en su casa, se enojan, lo quieren matar, sacan el cuchillo, lo quieren matar más, sacan el revolver y lo guardaran. El director habla de una amiga que no sabe que rol cumple, pero baila, es locutora, pone las luces y es la que pedalea. Y hay otra, toca el violín . Hacen juegos del millón pero pierden entre ellos mismos, son versados en idiomas como el inglés, español, portugués y no mucho en el japones. Todos mueren como soldados aunque reviven tomando sol en un edificio; tocándose (Celia llegó a 15 en un día) Se pelan con una enciclopedia, tu concha es como baba de perro, le dicen, y vos sos un hotel, pero lleno de pijas, le contestan.
\\
  
  Te hacen taparte los ojos cuando escuches un sonido y  aparecen con una mesa, todos de gala y con un libro. El personaje-director lee una lista de posibles finales, como que un posible final era leer la lista (el 12) para después leer un  mail que mando a los personajes (20) diciendo lo largo del camino, lo encontrado en estas funciones y no resistirse al final, otro era que no tenga final (33). El director se despide y los actores haciendo de personajes se despiden de sus papeles, aplausos, se desmantelan las paredes y se vuele abrir la barra. Ahora soy yo el que estuvo ahí y con mi fanzine en una mesita les doy el monológo, dando razones para ir a ver una breve enciclopedia sobre la amistad, pero ya termino. 


\end{document}

%%--------------Termino del documento------------------------------------------------------
